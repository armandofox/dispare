\section{Builder}

The builder is a collection of shell scripts that work together to
transform the user's input into a compilable Chare Kernel program.  The
builder first reads the {\tt .spc} file and figures out the names of the
events and which input and state structures correspond to each event
type.  It then executes a series of scripts, which use standard Unix
filters to perform various kinds of
replacements on the prefabricated ``skeleton'' files.  In particular,
the following assumptions are made:

\begin{enumerate}
\item Standard (System V or later) versions of the stream editor {\tt
sed} and the pattern processor {\tt nawk} are available and are on the
default path.

\item The standard C compiler {\tt cc} is installed.  The preprocessor
(invoked with the {\tt -E} option) is used to prepare the user's files.
It is also assumed that the first line generated by the preprocessor is
line control information, which \dispare will throw away; this is
standard behavior for the preprocessor.

\item The builder creates a directory {\tt /tmp/dispare} and
manipulates various files in it.

\item A few temporary files, consisting of user file names with {\tt .c}
appended, are created in the current directory and removed when the
builder is finished.
\end{enumerate}

The skeleton files are ``least common denominator'' files which include
code common to all event types.  Parts of the code which will differ
depending on the event type are delimited by special characters; the
builder scripts look for these and replace them with the code
appropriate for a given event type.  The builder scripts are run once
for each event type; in each case they use the skeleton file and
information from the user's input files to construct a state or event
chare {\tt .p} file suitable for compilation, and a file containing the
event-specific Chare Kernel message definitions for that event.  

In the case of the event chare, the builder script moves the user's C
code directly into the event chare description.  The event scheduling
calls are replaced with a Chare Kernel {\tt SendMsg} call, as well as
code to initialize the various fields of the message being sent.

When all these files have been created, they are {\tt cat}'d together
into one large {\tt .p} file, since the current version of the Chare
Kernel does not support separate compilation.  The resulting {\tt .p}
file can be read by the Chare Kernel translator and subsequently
compiled using {\tt cc}.

Currently, the builder is composed of shell scripts which perform almost
no error checking.  Ideally, the builder should be more robust, and
ultimately written as a separate C program; but for the common case when
the input files are correct, the use of standard Unix filters greatly
simplified the builder.