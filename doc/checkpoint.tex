\section{State Chare With Checkpointing Strategy}


If speculative execution is allowed, then checkpointing needs to be
done.  Our checkpointing design of \dispare uses the same Event Chare
used for our test case, only the State is different.  Although our
checkpointing State Chare design is much more complicated, it only
requires two more inter-chare messages, {\em Revert} and {\em State GTS}
(see figure \ref{figstatestate}).  If a State Chare discovers that it
should not have performed some speculative work, it uses the Revert
message to tell its children to ignore the input messages sent.  The
State GTS message is used to determine the {\em Event's global time step
(GTS)}.  It uses the GTS to determine the states and input messages that
no longer need to be saved.

\begin{figure}
\label{figstatestate}
\centerline{\psfig{figure=fig_state_state.ps}}
\caption{State Chare to State Chare Communication for Checkpointing}
\end{figure}

The Revert and Event GTS messages from one Event's State Chare to
another's must also be ordered, but the ordering
cannot be done independently.  The checkpointing messages must be
ordered with respect to the input messages and each other so that
checkpointing is applied to the correct input messages and states.

This is where our checkpointing design fails using the Chare
Kernal.  The Revert message would be sent by the State Chare to a
child State Chare.  It would then start execution of a new Event
Chare.  We had assumed that any input messages the Event Chare sent to
the child State Chare would reach the State Chare after the Revert
message and hense be processed after the Revert message.  Message
ordering would therefor be preserved.  Our assumption was based on the
fact that the Revert message travels directly from the State Chare to
the child State Chare while the input message is dependent on a chain
of messages, a State Chare to Event Chare message and then and Event
Chare to child State Chare message.  We assumed that the worst case
senario is when the input message chain of events would travel the
same distance on the underlying architecture as the revert message
did.  Since the message dependency flow of the input message is
initiated after the revert message, the revert message would get to
the child State Chare first (and be processed first).  Note that we
also assumed that Revert and Event GTS messages would be processed by
a child State Chare in the same order that they were created because
they originated from the same State Chare.

After many hours of work on our checkpointing design, we discovered
that this assumption was false.  To overcome this problem our design
requires another layer of message buffering that would enforce message
ordering of the messages the State Chare received.  Message ordering
needs to be enforced for different messages across multiple entry
points with the messages coming from both the parent Event's State and
Event Chares.  But the messages from the State and Event Chares of
different parents need to be ordered independently of each other.

Our design, assuming that this extra layer of message buffering is
implemented, is now described in detail.

\subsection{Input History}

The input values received by the State Chare are stored in the Input
History list, a linked list that grows and shrinks as needed.  This
list is used to keep a history of the input values received.  Its
format is shown in figure \ref{figinhis}.

\begin{figure}
\label{figinhis}
\centerline{\psfig{figure=fig_input_his.ps}}
\caption{Input History List}
\end{figure}

Current is a pointer that points to the top of the list of inputs that
have yet to be processed by the Event Chare.  When the State Chare
receives a new input from parent N, it inserts the input with the
parent number, parent's time step and the input's time step into the
list.  Current is updated if needed so that the new input is included
in the list of inputs to be processed.

Input values are deleted from the list if their time step is less than
the Event's global time step.  The calculation of the Event's global
time step is discussed in section \ref{globaltime}.

\subsection{State History}

The state history is stored in the State History array, an array of
fixed size.  Its format is shown in figure \ref{figstatehis}.

\begin{figure}
\label{figstatehis}
\centerline{\psfig{figure=fig_state_his.ps}}
\caption{State History Array}
\end{figure}

The Save State field stores the state of the Event after the Event
Chare has processed the inputs for the time step indicated in the Time
Step field.  Begin and End are pointers that define the range of valid
saved states in the array.  The save states are organized in order of
their time step with the youngest being at Begin and the oldest at
End.  Whenever the Event Chare finishes processing for time step T,
the Event's state is saved by adding it to the State History array.

Saved states are removed in two ways.  Any saved states with time
steps less then the Event's global time step is removed from the array
by updating Begin.  If a revert message or an input message is
received with a time step less than the time step at End, all saved
states with time steps greater than the revert's or input's time step
are deleted by updating End.

\subsection{Initiating Event Chare Processing}

Whenever an Event is ready for firing, the State chare uses the inputs,
with time step T, indicated by the Current pointer.  The State
Chare takes all the inputs on the Input History list with time step T,
takes the state that End points to on the State History, and sends
this along with T and any default input values needed to the Event
Chare.  If the State History array is full, the send is stalled until
some states are removed.  If the Event is asynchronous, the send is
stalled until all inputs fro the time step T have arraived.  For a
monitor Event, the send is stalled until the Event's global time step
equals T.  After the send, the Current pointer is updated.

\subsection{Reverting to Previous States}

If the State Chare receives an input with time step T that causes the
Current pointer to point to an older time step, then the State Chare
must revert its state to that earlier time step.  It does this by
removing all states from its state history with time steps $\geq$ T.  The
State Chare must also tell all its children that inputs received from
its Event are invalid for the Event's time steps $\geq$ T.  To do this the
State Chare sends a revert message of T to the State Chares of all of
the Event's children.

When a State Chare receives a revert message for time step R from a
parent chare N, it will eliminate all inputs from N with a source
Event's time step $\geq$ R.

If an input is eliminated with a time step T that is less then the
Current's time step then the State Chare must revert its state to that
earlier time step.  Current will be updated to include all inputs with
time steps $\geq$ T and all states with time steps $\geq$ T will be removed
from the state history.  Then the State Chare will send a revert
message with time T to the State Chares of all of the Event's children.

\subsection{The Event's Global Time Step}
\label{globaltime}

Associated with each Event is an Event's {\em global time step}, {\em
GTS}.  The global time step is used to remove unneeded checkpointed
values in the input and state histories.  The Event will receive no
new inputs values from any parent Event for a time step $\leq$ GTS and it
will produce no new input values to any child Event for a time step $\leq
GTS+1$.  The Event's GTS is calculated as follows:

$GTS = min (min(GTS_i + 1), ((Event's Active Time Step)-1))$

where $i$ is the entire set of parents for that Event.  The Event's GTS
is calculated from the minimum of the GTSs of all the Parent Events
and from the the time step that the Event is currently processing.  If
the Event is not processing any inputs and has no inputs to process,
its active time step is effectively infinite and is treated as such.

The State Chare will update its GTS whenever it receives a new GTS
from a parent or whenever its active state time step changes.  If this
GTS update results in a new GTS value, the State Chare sends its
$GTS+1$ to the State Chare of all its children.  It will also remove
any uneeded inputs and states from the input and state histories.

