
\section{Event Chare}
\label{evtchare}

\subsection{The User's View}

The Event Chare is the chare where the user's code for an instance of an
Event is executed.  The user writes in ordinary C code the work that is
initiated by its inputs for a single time step.  The variable TIME
contains the value of this time step.  Here is an important point that
must be stressed, once the user's code receives inputs for a cirtain
time step, it will not receive any new inputs.  The user's code must
finish execution and then it will be initiated with a set of inputs for
a following time step.  If the user wishes to retain the value of any
variable from one initiation to another, the user must store the value
in STATE.  STATE is a structure that the user can define for each Event.
Any value that is not stored in STATE will be lost from one initiation
to another.

An Event passes values to another Event through the use of the Link
variables.  A Link variable is named and defined by the user.  It is a
structure that contains all the values the users wishes to pass from one
Event to another.  If the user wishes to pass values from EventA to
EventB through the Link Link\_A\_B, the user would copy all the values
into Link\_A\_B and then use the following in his/her code:

{\tt SCHEDULE EventB (Link\_A\_B, $TIME + T$);}

T is the relative scheduled time step that the input will be processed
by EventB.  

An Event can be one of three types: synchronous, asynchronous, or
monitor.  {\em Synchronous} means that all input values must be
received from all Links for a time step before the inputs can be
processed by the Event.  {\em Asynchronous} means the input links can
be processed for a time step even though values have not been received
from all input Links.  The event chare will process the Event with the
input values it has received and will use user defined defualts for
the values of the other inputs.  A {\em monitor event} is an Event
with no output Links.  It is used by the user to monitor the
simulation while the simulation is running.

For our simulation test case (see Chapter \ref{example}), we limited the
Events to synchronous inputs.

From the users viewpoint, this is all that needs to be known.  The
user's view of the \dispare event simulator is a group of event nodes
interconnected by Links.

\subsection{Implementation on the Chare Kernel}

Regardless of the what type of Event it is (synchronous, asynchronous,
or monitor) the Event Chare is the same.  The Event Chare skeleton is
shown in the appendix.

The Event Chare has two entry points.  The first is an initialization
entry point called {\tt Init}.  In the {\tt Init} entry point the
Event Chare receives the chare IDs of all the State Chares from its
State Chare.  It also initiallize all the variable it needs.  The
second entry point is {\tt Event}.  The State Chare sends a messages
to this entry point containing all the inputs for a certain time step,
the most resent state of the Event, and the time step for the inputs.
The Event Chare copies this information into local variables and then
allows the user's code to run.  The user's code will then process the
inputs for that time step.  The {\tt Event} entry point will then copy
the new state of the Event into a message and send it to the {\tt
EVENT\_DONE} entry point of its State Chare.

At every point of the user's code where {\tt \#EVENT\_NAME\#(LINK,
TIME+T)} appears, it is replaced with a block of send message code
(see appendix).  This code copies the value of {\tt LINK}, {\tt TIME},
and {\tt TIME+T} into a message and sends it to the {\tt NEW\_EVENT}
entry point of the child Event {\tt \#EVENT\_NUM\#}.




