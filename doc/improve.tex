\section{Possible Improvements}
\label{improvements}

This section discusses possible improvements, both in the implementation
of \dispare and the Chare Kernel.

One possible improvement would be for the Chare Kernel to allow for
execution of the {\tt Event} entry point to sleep (stall) untill it
receives a response to a message.  (This would be used whenever the
Event Chare sent a NEW\_EVENT message to a child State Chare.)
Currently we can't do this since all the Chare Kernel calls are
non-blocking, i.e.  there is no receive primitive.  Another way to do
this is to allow a command that would stop execution of an entry point
and allow the chare to receive new input messages.  A ``return'' message
to the entry point would then allow execution to continue where it left off.
We feel this option would greatly improve the flexibility of the Chare
Kernel.  Right now the only way to receive new message information is at
the beginning of a C code block; but often there is a need to stop
and wait for additional message information when deep inside a code
block, and jumping into another entry point would be inconvenient.

This response message would be used by \dispare to insures that to many
inputs are not queued.  If a State Chare has too many input messages
from a parent Event and its input queue is full, it will not
respond to the parent Event.  The parent Event sleeps and allows other
Events to be processed until it receives the respond message.  This
helps make sure that input queues do not become excessively large with
inputs to be processed.  It also helps keep one Event type from monopolizing
processing time and distributes processing time among Events.  For
example, if an Event has a loop with many iterations, it can continually
generate input messages for other Events while monopolizing a
processor and not allow those other Events to process the inputs.
This can become a major problem if the dependency graph has a lot of
Event nodes and the simulation is being run on a machine with a very
small number of processors.

Another useful addition to the Chare Kernel would be the ability to
kill a Chare message execution before and/or while it is being
executed by an entry point.  Once the State Chare fires an Event, it
may receive a message that causes it to revert.  At that point is know
that any the work Event Chare is useless.  Killing execution of the
message sent to the Event Chare, or even killing it before execution
starts, would help increase efficiency.  This is especilly true if the
Event Chare has a very long execution time.  Eliminating any input
messages produced by the Event Chare would still be handled by sending a
revert message.


