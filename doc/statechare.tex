\section{State Chare}

The state chare for an event type is responsible for firing instances of
that event, passing it the correct state, inputs and timestep, and
updating the event state when an instance finishes.  There is a single
state chare instance for each event type.  All message scheduling is
done through the state chare.

For the purposes of this explanation, a {\em parent} event schedules a
{\em child} event (the event types may be the same or different), and
provides one input to that event.  An {\em event-scheduling message}
contains the value of the input provided by the parent, the timestep
$T_S$ at which the child event is scheduled to be executed, and an index
$i$ indicating which parent (event type) created the event.

\subsection{Functional Overview}

Associated with each Event Chare is a State Chare.  The State Chare
receives the inputs that are intended for its corresponding Event, and
initiates the execution of the Event Chare.  For checkpointing purposes
it also keeps a history of the STATE of the Event and of the inputs
received.  The figure \ref{figStateEvent}  shows the message
interconnections of the State Chare and Event Chare.

\begin{figure}
\label{figStateEvent}
\centerline{\psfig{figure=fig_state_event.ps}}
\caption{State and Event Chare Message Interaction}
\end{figure}

\subsection{No Checkpointing Implementation}

Our test case (see chapter \ref{example}) implements a basic parallel
event simulator without checkpointing.  To eliminate the need for
checkpointing the Events are restricted to synchronous inputs and
message ordering is enforced for input messages.

To insure that the input messages sent by the Event Chare are processed
by the State Chare in the same order they were created, a sequence
counter is used.  Augmented with each input message is a sequence
counter.  For every input message that the Event Chare sends,
the counter is incremented by one.  The State chare uses the sequence
counter in the message to insures that each input is processed in the same
order that they were created by the Event Chare, even if the messages are
received out of order, as can happen in message passing systems.

This enforcement of message ordering along with the restricted of
synchronous only Events eliminates speculative execution and the need
for checkpointing in our test case.  Speculative work would occur when an Event
will execute even though messages have not been received for all of
its inputs.  If those input messages never show up, then the execution
will be useful.  If those input messages do show up, then the
execution is invalid and the Event needs to revert back to a previous
state by checkpointing.  If an
Event is restricted to synchronous inputs, it will wait until it
receives messages for all its input before it executes.

The only additional need for checkpointing is if an Event receives a revert
message from a parent Event.  This can occur if the Event receives an input
message from a parent, executes it, and then is told to disregard the
input message by that same parent.  The only way this can occur
is if the parent needed to revert to a previous state (and undo all the
input messages to its children).  It becomes obvious that someone
needs to initiate the revert sequence; the only thing that can do so
is reverting due to speculative execution.

\subsection{Initialization}

The state chare has two initialization entry points; let us refer to
them as Init 1 and Init 2.
Initially, an instance of each state chare is created by the main chare
(see section \ref{overview}).  The Init 1 entry point in each
state chare sends its chare ID back to the main chare;  the main chare
constructs an array containing the chare ID's off {\em all} the state
chares.  Then, this array is sent to the Init 2 entry point in each
state chare.  The Init 2 entry point creates the single event chare
instance and passes it this array, and then schedules the first event at
time 0.  The Init 1 entry point is also used to receive information
concerning when the simulation should stop.

The reason for all this communication is that since each event can
potentially try to schedule other events of {\em any} type (not just its
own), it needs to know the chare ID's of all the state chares so it can
schedule all event types.

\subsection{New Event Scheduling}

An event-creation statement in the user's code (see chapter \ref{input})
is translated into a Chare Kernel {\tt SendMsg} call, in which an
event-scheduling message is sent to the state chare.  The state chare
maintains a queue of pending events.  Explicit message sequencing is
used to insure that event instances are fired in the correct order; the
queue is sorted by message sequence number, and messages are never
processed out of sequence number order.

Note that a new event cannot be fired until the last fired event of the
same type has completed.  This is because an event instance requires
access to the event state, and the state won't be known until a previous
instance has finished executing and modifying the state.

If this is the first scheduling request for this event (as determined by
searching the queue for an entry whose $T_S$ matches that in the
scheduling message), a new queue entry is created, and the input
provided by the parent that sent the scheduling message is installed in
the queue entry.  If a queue entry already exists with a matching
$T_S$, then the input specified in this message is installed in the
(previously default-valued) slot in that queue entry.  It is an error
for two parents to try to fill the same input slot for the same instance
of a child event; this condition is detected and reported  by \dispare.

If the state chare is currently {\em idle} (no event instance of its
type is currently executing), {\em and} if all of the inputs for the
scheduled event have arrived, the new event is fired; otherwise nothing
happens (i.e. the entry point terminates).  Firing an event consists of
sending a message to the already existing event chare, containing the
timestep, inputs, and current state associated with the event type; and
clearing the idle flag to indicate that a scheduled event is now executing.

\subsection{Event Completion}

When an instance of an event chare finishes executing, it sends a
state-update message back to its state chare (see section
\ref{evtchare}).  At this time, the state chare replaces its notion of
the current state of the event with the state structure contained in the
state-update message.  If there are events in the queue, the next
pending event is fired if all its inputs are available; if there are no
events in the queue, or if the next queued event is still waiting for
inputs, the idle flag is set so that event firing can occur the next
time an input is received.

\subsection{Stopping Simulation}

Whenever the main chare receives a StopSim message (see Chapter
\ref{input}), it broadcasts copies of the message to all the state
chares.  The state chare responds by updating its notion of when the
simulation should stop.  If the event type of this state chare has
already reached the specified timestep, no more events are fired.  If
all state chares behave this way, the result will be quiescence.  This
allows a normal exit through the quiescence EP even when the user has
used StopSim to abort the simulation.