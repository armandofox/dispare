\chapter{Builder Bugs and Warnings}
\label{syntax}

The following limitations are a result of the prototype-level
builder, and should go away once it is refined.

\begin{enumerate}

\item The {\tt dispare} variable in the build script must be correctly
set to point to the directory where \dispare files are.  This directory
should also be on the default search path.  The build script should be
run from the directory where the user's files are.

\item The user's specification file cannot contain any {\tt typedef
struct} declarations other than those for the input and state
structures.  Structures may be nested inside the input and state
structures, but the nested structures may not have a tag associated with
them. 

\item EVENT, INPUT, STATE and SCHEDULE are reserved words; the user may
not declare local or structure variables with these names.

\item In the {\tt .spc} files, the reserved words EVENT, INPUT and STATE
must appear on the {\em same line} as the identifier that follows them
(the structure tag).

\item Similarly, in the {\tt .evt} code files, the reserved word
SCHEDULE must appear on the same line as the event scheduling call.

\item No parenthesized items may appear as arguments to the event
scheduling call.  Simple expressions without parentheses, e.g. {\tt
TIME+delta-1}, are okay; but to be perfectly safe, use only identifiers.

\item Event names as declared by the user are reserved.  That is, if the
user declares an event named Foo1, no local variable or field within
another structure named Foo1 may exist, or the builder will get
{\em really} confused.

\item The event prototypes must follow the input and state structure
declarations.

\item Perhaps the worst limitation: there is literally no error checking
in the builder.  This is because the builder is implemented as a
series of shell scripts, so errors in the user file will simply result
in compile-time syntax errors or (in rare cases) compilable code that
doesn't execute as expected.
\end{enumerate}


